% This is LLNCS.DOC the documentation file of
% the LaTeX2e class from Springer-Verlag
% for Lecture Notes in Computer Science, version 2.4
\documentclass{llncs}

\usepackage{llncsdoc}
\let \proof \relax
\let\endproof\relax
\usepackage{amsthm,amsmath}
\usepackage{graphicx}

\usepackage{booktabs}
%\usepackage{subfig}
\usepackage{subfigure}
\usepackage{lipsum}
\usepackage{array}
\usepackage{algorithm}
\usepackage{algorithmic}
\usepackage{color,soul}
\usepackage{epstopdf}
%\newcolumntype{L}[1]{>{\raggedright\let\newline\\\arraybackslash\hspace{0pt}}m{#1}}
%\newcolumntype{C}[1]{>{\centering\let\newline\\\arraybackslash\hspace{0pt}}m{#1}}
%\newcolumntype{R}[1]{>{\raggedleft\let\newline\\\arraybackslash\hspace{0pt}}m{#1}}
\renewcommand{\algorithmicrequire}{\textbf{Input:}}
\renewcommand{\algorithmicensure}{\textbf{Output:}}
\newcommand{\algorithmicbreak}{\textbf{break}}
%\newtheorem{theorem}{Theorem}
%\newtheorem{corollary}{Corollary}
%\newtheorem{lemma}{Lemma}
%
\begin{document}

\title{Safe Multiclass Transfer Learning}
\maketitle
\begin{abstract}
	In transfer learning, domain adaptation tries to exploit the knowledge from a source domain with plentiful data to help learn a classifier for the target domain with a different distribution and little labeled training data. 
	In this paper, we investigate this problem under the setting of \textit{Hypothesis Transfer Learning} (HTL) where we can only access the source model instead of the data. We aim at two important issues: effectiveness of the transfer and compatibility of the target model with different types of source models in the HTL scenario and proposed our method, SMTLe.
	We illustrate that the feature augmentation strategy used in SMTLe can greatly increase the compatibility of the target model to fit different types of source models. To better exploit the source model, we use the bi-level optimization (BO) method to estimate the transfer parameter which measures the similarity of the source and target domains. We demonstrate that our BO problem is a strongly convex optimization problem and we can effectively obtain the optimal transfer parameter with the sub-gradient descent method. Empirical results show that SMTLe can effectively exploit the knowledge from different types of source models and outperform other HTL baselines as well.  
\end{abstract}

%%
\section{Introduction}
Domain adaptation for image recognition tries to exploit the knowledge from a source domain with plentiful data to help learn a classifier for the target domain with a different distribution and little labeled training data. In domain adaptation, the source and target domains share the same label but their data are drawn from different distributions.

In domain adaptation, the knowledge of the source domain can be transferred by 3 different approaches: \textit{instance transfer}, \textit{model transfer} and \textit{feature representation transfer} \cite{pan2010survey}. In this paper, we focus on the method that transfers knowledge from the source model. Some recent works show that exploiting the knowledge from the source model can boost the performance of the target model effectively\cite{tommasi2014learning}\cite{kuzborskij2013n}.
Moreover, in some real applications, we can only obtain the source models and it is difficult to access their training data for different reasons such as the data credential.   
Recently, a framework called Hypothesis Transfer Learning (HTL) \cite{kuzborskij2013stability} has been proposed to handle this situation. HTL assumes only source models (called the \textit{hypotheses}) trained on the source domain can be utilized and there is no access to source data, nor any knowledge about the relatedness of the source and target distributions. 


Previous research \cite{ben2010theory} \cite{ben2007analysis} shows that without carefully measuring the distribution similarity between the source and target data, the source knowledge could not be exploited effectively or even hurt the learning process (called  \textit{negative transfer})\cite{pan2010survey}. 
However, as we are not able to access the source data in an HTL setting, how to effectively and safely exploit the knowledge from the source model could be an important issue in HTL (Safety issue). Moreover, different source models can be trained with different kinds of classifiers. For example most models trained from ImageNet are deep convolutional neural networks while some models of the VOC recognition task could be SVMs or ensemble models. Therefore, a practical HTL algorithm should be compatible with different types of source classifiers (Compatibility issue). To the best of our knowledge, none of the previous work in HTL is able to solve these two issues at the same time.

In this paper, we propose our method, {called Safe Multiclass Transfer Learning (SMTLe)}, that can solve these two issues simultaneously. Previous work \cite{jie2011multiclass} suggests that feature augmentation can greatly increase the compatibility of the target model in the HTL scenario. To solve the compatibility issue, we propose a feature augmentation method that for each target example, we add its class probabilities from the source model as the auxiliary feature. Moreover, we apply different weights (called transfer parameters) for different auxiliary features to control the amount of the knowledge transferred from each source model. As a result, the value of the transfer parameter reflects the similarity between the source and target domain. By carefully estimating the transfer parameters, we can obtain the optimal target model.

To better estimate the transfer parameters, we treat them as the hyperparameter of a convex optimization problem and  
introduce bi-level hyperparameter optimization\cite{Pedregosa16} , which has been widely used for many different hyperparameter optimization problem, to estimate the optimal values. Specifically, on the low-level optimization problems, we use a least-square SVM to obtain the hyperplane and on the high level, we use the novel multi-class hinge loss with $\ell_2$ penalty. Different from many other bi-level optimization problems which are non-convex optimization problems, we show that our transfer parameter estimation problem is a strongly convex optimization problem and demonstrate that our method SMTLe can find the $O({\log(t)}/{t})$ optimal solution with $t$ iteration. 

In our experiment, we use the popular benchmarks Office and Caltech256 as our dataset. We show that SMTLe can successfully transfer the knowledge with different types of source models. Moreover, we show that our novel high level objective function with $\ell_2$ penalty can improve the performance of the target model effectively compared with SMTLe without $\ell_2$ penalty and other baselines in HTL. 

The rest of this paper is organized as follows: In Section \ref{sec:work} we introduce the issues in transfer learning and some related work regarding these issues.
In Section \ref{sec:prob}, we reformulate the HTL in Phase I and propose our method of feature augmentation. We show that we can better analyze the performance of the transfer learning algorithm with feature augmentation. Then, we propose a novel objective function for transfer parameter estimation, called SMTLe in Section \ref{sec:smitle}. We show that the estimated transfer parameter can evaluate the utility of the source hypothesis and alleviate negative transfer autonomously. In Section \ref{sec:exp}, we show the performance comparison between SMTLe and other baselines on a variety of experiments on MNIST and USPS datasets.


\section{Related Work}\label{sec:work}
The motivation of transferring knowledge between different domains is to apply the previous information from the source domain to the target one, assuming that there exists a certain relationship, explicit or implicit, between the feature space of these two domains \cite{pan2010survey}. Technically, previous work can be categorized into solving the following three issues: \textit{what}, \textit{how} and \textit{when} to transfer \cite{tommasi2014learning}.


\textbf{What to transfer.} Previous work tried to answer this question from three different aspects: (1) selecting transferable instances, (2) learning transferable feature representations and (3) transferable model parameters. Instance-based transfer learning assumes that part of the instances in the source domain could be re-used to benefit the learning for the target domain. Lim et al.\cite{lim2012transfer} proposed a method of augmenting the training data by borrowing data from other classes for object detection. Learning transferable features means to learn common features that can alleviate the bias of data distribution in the target domain. Recently, Long et al. \cite{LongICML15} proposed a method that can learn transferable features using deep neural network and showed some impressive results on the  benchmarks. 
A model transfer approach assumes that the parameters of the model for the source task can be transferred to the target task. Yang et al. \cite{yang2007cross} proposed Adaptive SVMs transferring parameters by incorporating the auxiliary classifier trained from the source domain. 
In addition to Yang's work, Ayatar et al. \cite{aytar2011tabula} proposed PMT-SVM that can determine the transfer regularizer automatically according to the target data. 
Tommasi et al. \cite{tommasi2014learning} proposed Multi-KT that can utilize the parameters from multiple source models for the target classes .
Kuzborskij et al. \cite{kuzborskij2013n} proposed a similar method to learn new categories by leveraging the known source models.

\textbf{When and how to transfer.} The question \textit{when to transfer} arises when we want to know if the information acquired from the previous task is relevant to the new one (i.e. in what situations knowledge should not be transferred). 
\textit{How to transfer} the prior knowledge effectively should be carefully designed to prevent inefficient and negative transfer. Previous work \cite{davis2009deep,wang2014active,zhou2014multi} has used the generative probabilistic method. Bayesian learning methods can predict the target domain by combining the prior source distribution to generate a posterior distribution. Alternatively, max margin methods\cite{kuzborskij2013n,tommasi2010safety,yang2007cross} try to use the hyperplane parameter to transfer the knowledge between source and target domains. Luo et al. \cite{jie2011multiclass} proposed MKTL used feature augmentation method to leverage the source model.

Our work corresponds to the context above. In this paper, we propose EMTLe based on the model transfer approach. In our case, we focus on how to exploit the knowledge from the prediction of the source models instead of using the parameters of the model to transfer the knowledge.


\section{Using the Source Knowlege as the Auxiliary Bias}\label{sec:prob}
In this section, we introduce our strategy in SMTLe that can exploit the knowledge from different types of classifiers. In general, for each example in the target domain, we use its output class probabilities from the source models as the auxiliary bias term to adjust the final output.

%\subsection{Auxiliary Bias from the source model}
To make our method compatible with different types of source model, we have to find a solution to align the knowledge of different types of classifier. It is clear that most of the existing classifiers can output the class probability for the input examples. Therefore, in SMTLe, we use the class probability of the target examples from the source model to align the knowledge of the different types of source model.  

To use the aligned source knowledge, we use a straight-forward way, by using the class probability as the auxiliary information to adjust the output of the target model (see Figure \ref{fig:ab}).
However, as we know that due to the domain shift, the performances of the different source models vary on the target domain. Therefore, to compensate this domain shift, we apply different weights to different source model outputs. Here, the weight of each model reflects the relatedness between the source model and the target domain. The more related they are, the larger weight we should apply. Specifically, in this paper, we call it \textit{transfer parameter}. Therefore, for any target learning $D_T=\{x,y\}$ and the given source model $f'$, our goal is to find the target model $f$:
\begin{equation}\label{eq:low_opt}
f=\underset{f \in \mathcal{F}}{\arg \min}\ell\left(f+\beta f',D_T\right)
\end{equation} 
where $\beta$ is the transfer parameter and $\ell(\cdot,\cdot)$ is the loss function to learn the target model.

\begin{figure}\label{fig:ab}
	\centering
	\includegraphics[scale=0.6]{fig/ab.png}
	\caption{Demonstration of using the source class probability as the auxiliary bias to adjust the output of the target model.}
\end{figure}
%Actually, using the output of the classification model as the auxiliary information to train another model has been widely used in different learning scenarios. In Vapnik's \textit{Teacher-student} diagram, it is represented as the \textit{privileged knowledge} and in Hinton's \textit{Distillation} diagram \cite{hinton2015distilling}, it is called \textit{soft-labels}.
%Different from those previous works where the model used to general the auxiliary features is trained from the same domain, our model is trained from a different but related domain.

There are several advantages of our feature augmentation with class probability: (1) It is an effective and easy way to align the knowledge from different types of source model.
(2) Features are naturally normalized in the same dimension as the class probability is always in the interval $[0,1]$. (3) The bonus advantage: increase the selection of the source domain. As SMTLe can select more types of source model, we have more options to select our source domain. As long as the model has the knowledge of the class we want (even though we may only need knowledge of one class in a multi-class classifier), we can still exploit the knowledge.

From Eq. \eqref{eq:low_opt} we can see that, once we can determine the value of the transfer parameter $\beta$, we are able to find the target model $f_T$. In the next part, we will show how we can effective estimate the unbiased transfer parameter effectively.







\section{Bi-level Optimization for Transfer Parameter Estimation}\label{sec:smitle}

In Eq. \eqref{eq:low_opt}, we have to find the optimal target function $f$ that minimize the training error on the target domain in addition with the source model $f'$ and the transfer parameter $\beta$. As we discussed before, the transfer parameter in Eq. \eqref{eq:low_opt} is a hyperparameter that is decided by the relatedness between the source model and target domain. A simple way to measure the relatedness is to evaluate the performance of the source model on the target training data. However, this estimate could lead to a relatively large variance when the target training data is small. In this paper, we use the leave-one-out cross-validation (LOOCV) strategy to reduce this variance. Previous research \cite{kuzborskij2013stability} suggests that LOOCV can increase the robustness of the estimated hyperparameter especially on small dataset. For the $i$-th round, the target set for training $D_T=\{x_i,y_i|i \in 1,...,l \}$ is split into training set $D_T^{\backslash i}=\{x_j,y_j|j\neq i\}$ and validation set $D_T^{i}=\{x_i,y_i\}$. The transfer parameter can be optimized with the following bi-level optimization (BO) function:
\begin{equation}\label{eq:BO}
\begin{aligned}
&\underset{\beta}{\arg \min}\sum_i^l\mathcal{L}(f^{i}(\beta),D_T^{i})\\
f^{i}(\beta)&=\underset{f \in \mathcal{F}}{\arg \min}\ell\left(f+\beta f',D_T^{\backslash i}\right) 
\end{aligned}
\end{equation} 
Here, we can use any convex objective function (e.g. SVM objective function) as $\ell(\cdot,\cdot)$ for the low-level optimization problem and $\mathcal{L}(\cdot,\cdot)$ is our the high-level cross-validation objective function.
In many previous works\cite{maclaurin2015gradient}\cite{Pedregosa16}, BO optimization is a non-convex problem and can only obtain the approximate solution. However, in our paper, we will show that problem \eqref{eq:BO} is strongly convex and we are able to obtain its optimal solution. 
\subsection{Low-level optimization problem using mean square loss}
To better illustrate our learning scenario, define our learning process as follows. Suppose we have $N$ visual categories and 
can obtain $N$ source binary classifiers $f'=\{f'_1,...,f'_N\}$ from the source domain. We want to train a target function $f$ consists of $N$ binary classifier $f=\{f_1,...,f_N\}$ using the target training set $D_T$ and the source models $f'$.
Specifically, in our BO problem Eq. \eqref{eq:BO}, for the low level optimization problem, we consider the scenario where each of the binary target model is a linear classifier $f_i = w_ix+b_i$.
Fand we use Least-square loss as the objective function to optimize each target model $f_n$ for any given transfer parameter $\beta$:
\begin{equation}\label{eq:bo_low}
\begin{aligned}
\text{Low-level:}\quad&f^{\backslash i}(\beta) : \min \sum_n^N\frac{1}{2}||w_n||^2+\frac{C}{2}\sum_j^l\left(Y_{jn}-f_n(x_j)-\beta_n f_n'(x_j)\right)^2\\
&\text{s.t.} \qquad x_j \in D_T^{\backslash i}
\end{aligned}
\end{equation}
Here, $Y$ is an encoded matrix of $y$ using one-hot strategy where $Y_{in} =1$ if $y_i=n$ and 0 otherwise.

The reason why we use the objective function \eqref{eq:bo_low} is that, it can provide an unbiased closed form Leave-one-out error estimation of each binary model $f_n$\cite{cawley2006leave}. 
Let $K(X,X)$ be the kernel matrix and
\begin{equation}\label{eq:linear}
\psi=\left[ 
{K(X,X) + \frac{1}{C}{\rm{I}}} \right]
\end{equation}
Let $\psi^{-1}$ is the inverse of matrix $\psi$ and  $\psi_{ii}^{-1}$ is the $ith$ diagonal element of $\psi^{-1}$. $\hat{Y}_{in}$, the LOO estimation of binary model $f_n$ for sample $x_i$, can be written as:
\begin{equation} \label{eq:loo}
{\hat Y_{in}} = {Y_{in}} - \frac{{{\alpha _{in}}}}{{\psi_{ii}^{ - 1}}}\quad {\text{for}}\quad n = 1,...,N
\end{equation}
where the matrix $\boldsymbol{\alpha}=\{\alpha_{in}|i=1,...l;n=1,...,N\}$ can be calculated as:
\begin{equation}
\boldsymbol{\alpha} =\psi^{-1} Y - \psi^{-1} f'(X)\boldsymbol{\beta}^T
\end{equation}

\subsection{High-level optimization problem using multi-class hinge loss with $\ell_2$ pentalty} 
For high level optimization problem, we use multi-class hinge loss\cite{crammer2002algorithmic} with $\ell_2$ penalty as our objective function.
\begin{equation}\label{eq:bo_high}
\begin{aligned}
\text{High-level:}\quad&\beta: \min \frac{{{\lambda}}}{2}\sum\limits_{n = 1}^N {{{\left\| {{\beta _n}} \right\|}^2}}  + \sum\limits_{i,n}\left[ {1 - {\varepsilon _{n{y_i}}} + {{\hat Y}_{in}} - {{\hat Y}_{i{y_i}}} - {\xi _i}} \right]\\
&\text{s.t.} \qquad1 - {\varepsilon _{n{y_i}}} + {\hat Y_{in}} - {\hat Y_{i{y_i}}} \le {\xi_i}
\end{aligned}
\end{equation}
Here, $\varepsilon _{n{y_i}}=1$ if $n=y_i$ otherwise 0.
Compared to the previous work \cite{tommasi2014learning}\cite{kuzborskij2013n} which uses the multi-class hinge loss without the $\ell_2$ penalty, there are two main advantages for our objective function: (1) When the training set is small, our LOOCV estimation could have a large variance. Similar to the penalty term in low-level problem \eqref{eq:bo_low},  the $\ell_2$ penalty here can {reduce this variance and improve the generalization ability of the estimated transfer parameter}. (2) With the $\ell_2$ penalty, optimization problem \eqref{eq:bo_high} become a strongly convex optimization problem w.r.t. the transfer parameter $\beta$. Therefore, we can obtain an $O({\log(t)}/{t})$ optimal solution with $t$ iterations using Algorithm \ref{alg:1} (see proof in Theorem \ref{th:1} in Appendix).
\input{agl1.tex} 
%In this section, \hl{we focus on Phase II of our framework, estimating the transfer parameter. We introduce an algorithm, called SMTLe, that can effectively estimate unbiased transfer parameter from a small training set and alleviate negative transfer}. 

\subsection{Leave-One-Out estimation of LS-SVM }
In the previous section, we introduce a novel perspective for HTL by feature augmentation. We show that how to set the values of the transfer parameters can significantly affect the performance of the target model. To achieve better performance of the target model, we have to reduce the training error and limit the VC dimension of the target model to improve its performance and alleviate negative transfer. In this part, we introduce the Leave-One-Out cross-validation  (LOO-CV) error to estimate the training error of each target binary model.


As we discussed above, we have to choose the proper transfer parameters $\boldsymbol{\beta}$ to minimize the empirical risk on the target training set to exploit the source knowledge.
In this paper, we choose the Leave-One-Out (LOO) cross-validation error to estimate the empirical risk for the following reasons: (1) It is proven that LOO error has a low bias on small training data regime \cite{kuzborskij2013stability}. The Leave-One-Out error is an almost unbiased estimator of the generalization error \cite{elisseeff2003leave}. (2) For LS-SVM, we can obtain unbiased LOO-CV error in closed form which means we can estimate the values of the transfer parameters in a more efficient way.

Let $K(X,X)$ be the kernel matrix and
\begin{equation}\label{eq:linear}
\psi=\left[ 
{K(X,X) + \frac{1}{C}{\rm{I}}} \right]
\end{equation}
The unbiased LOO estimation for sample $x_i$ can be written as \cite{cawley2006leave}:
\begin{equation} \label{eq:loo}
{\hat Y_{in}} = {Y_{in}} - \frac{{{\alpha _{in}}}}{{\psi_{ii}^{ - 1}}}\quad {\text{for}}\quad n = 1,...,N
\end{equation}
Here $\psi^{-1}$ is the inverse of matrix $\psi$ and  $\psi_{ii}^{-1}$ is the $ith$ diagonal element of $\psi^{-1}$. 

Let $F'(X)=\left[f'_1(X),...,f'_N(X)\right]$ be the output matrix of the source models and define $\begin{array}{c}\boldsymbol{\alpha'} \end{array}$ and $\begin{array}{c}\boldsymbol{\alpha}''\end{array}$ as follow:
\begin{equation}
\begin{array}{cc}
\boldsymbol{\alpha'} =\psi^{-1} \times Y & \boldsymbol{\alpha''} =\psi^{-1} \times F'(X)
\end{array}
\end{equation}

The matrix $\boldsymbol{\alpha}=\{\alpha_{in}|i=1,...l;n=1,...,N\}$ in Eq. \eqref{eq:loo} can be calculated as:
\begin{equation}\label{eq:solution}
 \boldsymbol{\alpha}  = \boldsymbol{\alpha} ' - \boldsymbol{\alpha} ''\boldsymbol{\beta ^T}
\end{equation}

Now, we already have an effective way to measure the performance of each target binary model with different $\boldsymbol{\beta}$ for our task. In the next subsection, we expand it to the multi-class scenario to estimate the optimal transfer parameters.

\subsection{Loss Function of SMTLe}
In this subsection, we propose a novel objective function according to our multi-class prediction loss function for transfer parameter estimation. We show that we can effectively obtain the optimal  $\boldsymbol{\beta}$ that alleviates negative transfer. 

For the multi-class scenario, we use One-versus-all strategy to assign the label to class $j$ if $j \equiv \arg {\max _{n = 1,...,N}}\left\{{f_n}(x)\right\}$. Let us call $\xi_i$ the multi-class prediction error for example $x_i$. $\xi_i$ can be defined as \cite{crammer2002algorithmic}:
\begin{equation}\label{eq:train_loss}
\xi_i(\beta) = \mathop {\max }\limits_{n \in \left\lbrace 1,...,N \right\rbrace } {\left[ {1 - {\varepsilon _{n{y_i}}} + {{\hat Y}_{in}}\left( {\beta_n } \right) - {{\hat Y}_{i{y_i}}}\left( {\beta_{y_i} } \right)} \right]}
\end{equation}
Where $\varepsilon _{n{y_i}}=1$ if $n=y_i$ and 0 otherwise. The intuition behind this loss function is to enforce the distance between the true class and other classes to be at least 1. 
 
From Eq. \eqref{eq:train_loss} we can see that, different from the binary scenario where 0 is used as the hard threshold to distinguish the two classes, our multi-class loss only depends on the gap between the decision function value of the correct label ($\hat Y_{y_i}$) and the maximum among the decision function value of the other labels ${{\hat Y}_{in}}(n \ne y_i)$. To reduce $\xi_i$ for a specific example $x_i$, we only have to increase the gap between ${{\hat Y}_{in}(n \ne y_i)}$ and ${{\hat Y}_{i{y_i}}}$. 

Instead of optimizing $\xi_i$ directly, we add the extra regularization terms for $\boldsymbol{\beta}$. Then we define our objective function as:
\begin{equation}\label{eq:loss}
\begin{aligned}
& \textbf{min}
& & \frac{{{\lambda}}}{2}\sum\limits_{n = 1}^N {{{\left\| {{\beta _{n}}} \right\|}^2}}  + \sum\limits_{i = 1}^l {{\xi _i}}   \\
& \textbf{s.t.}
& & 1 - {\varepsilon _{n{y_i}}} + {\hat Y_{in}}\left( {\beta_n } \right) - {\hat Y_{i{y_i}}}\left( {\beta_{y_i} } \right) \le {\xi_i};\\
& & &\lambda_1,\lambda_2 \ge 0
\end{aligned}
\end{equation}

Here $\lambda$ is the regularization parameter. This objective function can improve the performance of the target model on the unseen test data from two aspects: improve the generalization ability by limiting the VC dimension and reduce the empirical risk compared to no transfer model.

As we discussed in Section \ref{sec:prob}, regularizing the transfer parameters could improve the performance of the target model. Moreover, by adding the regularization term, the objective function \eqref{eq:loss} turns to be strongly convex. Therefore, we are able to use sub-gradient descent \cite{boyd2004convex} to guarantee its convergence to be $\mathcal{O}(\log(t)/t)$ optimal in $t$ iterations  (see proof in Appendix \ref{appd:convg}). This promises we can find the optimal transfer parameters effectively.
We can also show that this objective function can achieve lower empirical risk compared to no transfer model (see Appendix \ref{appd:proof}). This is very important when the source and target domains are not very related.

 
%\subsection{Optimizing the transfer parameter}
By adding a dual set of variables in objective function \eqref{eq:loss}, one for each constraint in, we get the Lagrangian of the optimization problem:
\begin{equation}\label{eq:dual}
\begin{aligned}
 &L\left( {\beta ,\xi ,\eta } \right) =
 \frac{{{\lambda}}}{2}\sum\limits_{n = 1}^N {{{\left\| {{\beta _n}} \right\|}^2}}  + \sum\limits_{i = 1}^l {{\xi _i}} \\
   &+ \sum\limits_{i,n} {{\eta _{i,n}}\left[ {1 - {\varepsilon _{n{y_i}}} + {{\hat Y}_{in}}\left( {\beta_n } \right) - {{\hat Y}_{i{y_i}}}\left( {\beta_{y_i} } \right) - {\xi _i}} \right]}  \\
 &\textbf{s.t.} \quad  \forall i,n \quad {} {\eta _{i,n}} \ge 0
\end{aligned}
\end{equation}

To obtain the optimal values for the problem above, we introduce our method using sub-gradient descent \cite{BoydCO} and summarize it in Algorithm. \ref{alg:1}. 
\input{agl1.tex}


\section{Experiment}\label{sec:exp}
In this section, we show empirical results of our algorithm for different transferring situations on two image benchmark datasets: Office and Caltech.
\subsection{Dataset \& Baseline methods}
Office contains 31 classes from 3 subsets (Amazon,Dslr and Webcam) and Caltech contains 256 classes. We select 13 shared classes from two datasets\footnote{13 classes include: backpack, bike, helmet, bottle, calculator, headphone, keyboard, laptop, monitor, mouse, mug, phone and projector}. The input features of all examples are extracted using AlextNet\cite{krizhevsky2012imagenet}.
%Because the two subsets Dslr and Webcam are relatively small and don't have data for testing, we only use them as the source domain.
\begin{table}[htbp]
	\centering
	\caption{Statistics of the datasets and subsets}
	\begin{tabular}{|c|c|c|c|c|}
		\hline
		Dataset&Subsets&\# classes &\# examples & \# features\\\hline
		\multirow{3}{*}{Office} & Amazon &13&1173 & 4096\\
		
		& Dslr &13&224 & 4096\\
		& Webcam &13&369 & 4096\\
		\hline
		Caltech256&Caltech&13&1582&4096\\
		\hline
	\end{tabular}%
	\label{tab:class_info}%
\end{table}%
We compare our algorithm EMTLe with two kinds of baselines. The first one is the methods without leveraging any source knowledge (no transfer baselines), including two methods. \textbf{No transfer:} SVMs trained only on target data. Any transfer algorithm that performs worse than it suffers from negative transfer. \textbf{Batch:} We combine the source and target data, assuming that we have full access to all data, to train the SVMs. The result of the Batch method is expected to outperform other methods under the HTL setting as it can access the source data. The second kind of baseline consists of two previous transfer methods in HTL, \textbf{MKTL\cite{jie2011multiclass}} and \textbf{Multi-KT\cite{tommasi2014learning}}. Similar to EMTLe, both of them use the LOOCV method to estimate the relatedness of the source model and target domain, but they use their own convex objective function without the $\ell_2$ penalty terms. We use linear kernel for all methods in all our experiments.
\subsection{Transfer from Single Source Domain}
In this subsection, following the protocol in \cite{jie2011multiclass,tommasi2014learning} for fair comparison, we perform 12 groups of experiments under the setting of HTL. 
For each experiment, one of the 4 (sub)datasets is selected as the source, while another dataset is used as the target. We evaluate the the effectiveness of EMTLe when all source models are of the same type (linear SVMs).
The size of each target dataset is varied from 1 to 5 to see how EMTLe and other baselines behave under the extremely small dataset.
We perform each experiment 10 times and report the average result in Figure \ref{fig:exp}. 
\begin{figure}[th]
\centering
\subfigure[C$\rightarrow$A]{
    \includegraphics[width=0.22\textwidth]{fig/caltechtoamazon.png}\label{a}
}
\subfigure[D$\rightarrow$A]{
    \includegraphics[width=0.22\textwidth]{fig/dslrtoamazon.png}\label{b}
}
\subfigure[W$\rightarrow$A]{
	\includegraphics[width=0.22\textwidth]{fig/webcamtoamazon.png}\label{c}
}
\subfigure[A$\rightarrow$C]{
	\includegraphics[width=0.22\textwidth]{fig/amazontocaltech.png}\label{d}
}\\
\subfigure[D$\rightarrow$C]{
	\includegraphics[width=0.22\textwidth]{fig/dslrtocaltech.png}\label{e}
}
\subfigure[W$\rightarrow$C]{
	\includegraphics[width=0.22\textwidth]{fig/webcamtocaltech.png}\label{f}
}
\subfigure[A$\rightarrow$D]{
	\includegraphics[width=0.22\textwidth]{fig/amazontodslr.png}\label{g}
}
\subfigure[C$\rightarrow$D]{
	\includegraphics[width=0.22\textwidth]{fig/caltechtodslr.png}\label{h}
}\\
\subfigure[W$\rightarrow$D]{
	\includegraphics[width=0.22\textwidth]{fig/webcamtodslr.png}\label{i}
}
\subfigure[A$\rightarrow$W]{
	\includegraphics[width=0.22\textwidth]{fig/amazontowebcam.png}\label{j}
}
\subfigure[C$\rightarrow$W]{
	\includegraphics[width=0.22\textwidth]{fig/caltechtowebcam.png}\label{k}
}
\subfigure[D$\rightarrow$W]{
	\includegraphics[width=0.22\textwidth]{fig/dslrtowebcam.png}\label{l}
}
\caption{Recognition accuracy for HTL domain adaptation from a single source. 5 different sizes of target training sets are used in each group of experiments.}
\label{fig:exp}
\end{figure}

\textbf{Observation \& discussion:} EMTLe can significantly outperform other baselines especially with a small training set. %Moreover, in some groups of experiments, they even suffer from negative transfer on the small training set. 
As we have discussed above, when the training set is small, with the transfer parameter estimated by our $\ell_2$ penalty in our high-level objective functions, EMTLe has a strong generalization ability and performs better on the test data. As the training size increases, the variance of training data decreases and the affect of the $\ell_2$ penalty term become less significant. Therefore, EMTLe and the other two HTL baselines show similar performance. 
It is interesting to see that MKTL even falls into negative transfer even with 5 training examples per class in some experiments. We found that, MKTL is more sensitive to the variance of the training data. Its performance is not as stable as Multi-KT and EMTLe over the 10 experiments. Because MKTL needs to learn more hyperparameters than Multi-KT and EMTLe, even though the training size increases, it may not be able to obtain a good model. 
In some experiments, we can see that EMTLe can even outperform the Batch method which can access more information and is expected to outperform the other methods under the setting of HTL.

\subsection{Transfer from Multiple Source Domains}
As we mentioned, EMTLe can exploit knowledge from different types of source classifiers which could greatly extend our selection of the source domain under the HTL setting. In this subsection we show that EMTLe can successfully transfer the knowledge from two source models of different types of source classifiers. Meanwhile, MKTL is used as our baseline which is also compatible with different types of source classifiers. 

In this experiment, we assume that there is no single source domain that can cover all classes in our target domain and we have to select source models from different source domains. Specifically, the 13 binary source models are selected from two different domains separately (6 from DSLR and 7 from Webcam) according to Table \ref{tab:class_gen}. Similar to our previous experiment configurations, we only use Caltech and Amazon as the target domains. We show the experiment results in Figure \ref{fig:exp2}.
% Table generated by Excel2LaTeX from sheet 'Sheet1'
\begin{table}[htbp]
	\centering
	\caption{The selected classes of the two source domains and the classifier type of the source model.}
	\begin{tabular}{|c|c|c|}
		\hline
		& class & classifier\\
		\hline
		DSRL& monitor,bike, helmet,calcu,headphone,projector & Logitic\\\hline
		Webcam&keyboard,mouse,phone,backpack,mug,bottle,laptop&SVMs\\ \hline
		
	\end{tabular}%
	\label{tab:class_gen}%
\end{table}%
\begin{figure}[th]
	\centering
	\subfigure[Dslr+Webcam $\rightarrow$ Amazon]{
		\includegraphics[width=0.4\textwidth]{fig/multi_amazon.png}\label{a2}
	}\qquad\qquad
	\subfigure[Dslr+Webcam $\rightarrow$ Caltech]{
		\includegraphics[width=0.4\textwidth]{fig/multi_caltech.png}\label{b2}
	}\\
	\caption{Recognition Accuracy for Multi-Model \& Multi-Source experiment on two target datasets. }
	\label{fig:exp2}
\end{figure}

\textbf{Observation \& discussion:} Under our multi-source scenario, it is more difficult to leverage the knowledge from the source models as the models are trained from different domains. From the results we can see that, in this complex situation, EMTLe can still transfer the knowledge from the source models despite the type of the source classifiers while MKTL can hardly leverage the source knowledge. EMTLe uses a simple way to leverage the source models and BO can help us better estimate the transfer parameter. However, MKTL uses a sophisticated feature augmentation to leverage the source models and has more hyperparameters to estimate. With a few training data, it is difficult for MKTL to measure the importance of each source model and exploit the knowledge from the models effectively.






\section{Conclusion}
In this paper, we present a novel method called SMTLe that is able to transfer knowledge of the source model in domain adaptation. Inspired by previous work,
we propose a novel perspective on the work of HTL and show the reasons why positive and negative transfer would happen in the different scenarios. Based on our analysis, we propose our method SMTLe that can safely leverage the knowledge from the source models to achieve the improved target model performance by limiting the VC dimension of the transfer problem and reduce the empirical risk as well. Experiment results show that SMTLe can leverage related source knowledge and alleviate negative transfer in different scenarios and outperform other baseline methods.

In our perspective on the domain adaptation problem, the feature augmentation approach can fit a wider range of source classifiers. We can leverage the knowledge from any source model that can output the decision score/confidence, such as the Neural Networks and the inference model. 
Meanwhile, there are still many open issues to solve before we could maximize the utility of different kinds of source classifiers. For example, how to better exploit the knowledge from a deep neural network with our feature augmentation framework and achieve good positive transfer performance and avoid negative transfer simultaneously. These challenges could lead to our future interest.




%\appendices
\section*{Appendix}
%\subsection*{Convergence Analysis}
\input{converge.tex}\label{appd:convg}

%\subsection*{Reduced training error}\label{appd:proof}
%Assume that $\bar \xi_i$ is the multi-class loss of example $x_i$ without utilizing any prior knowledge, i.e. $\beta = \mathbf{0}$. Let $ \beta^*$ be the optimal solution for Eq. \eqref{eq:dual} and $\xi_i^*$ be the multi-class loss with respective to example $x_i$. Then for every example $x_i \in \mathcal{X}$, we have:\[\sum\limits_i {{\xi^* _i}}  \le \sum\limits_i {{{\bar \xi }_i}} \]

\begin{proof}
When $\mathbf{\beta} = \mathbf{0}$, from Eq. \eqref{eq:train_loss} we can get:
\begin{equation*}
{\bar \xi _i} = \mathop {\max }\limits_n \left[ { {\varepsilon _{n{y_i}}}-1 + \frac{{\left( {{{\alpha '}_{i{y_i}}} - {{\alpha '}_{in}}} \right)}}{{\psi _{ii}^{ - 1}}}} \right]
\end{equation*}
For simplification, let $\delta_i=1$ if $i=N+1$ and 0 otherwise, and  ${\theta _{ij}} = {\alpha ''_{ij}}\left( {1 - {\delta _j}} \right)/\psi_{ii}^{ - 1}$.
To find the minimum of the primal problem, we require:
\begin{equation}
\frac{{\partial L}}{{\partial {\xi _i}}} = 1 - \sum\limits_n {{\eta _{in}}}  = 0 \Rightarrow \sum\limits_n {{\eta _{in}}}  = 1
\end{equation}   
\begin{eqnarray}\label{eq:opt_beta}
\frac{{\partial L}}{{\partial {\beta _n}}}  = 0 
\Rightarrow \beta _n^* = \frac{1}{{{\lambda}}}\sum\limits_{i,n} {\frac{{{\eta _{in}}{{\alpha ''}_{in}}}}{{\psi _{ii}^{ - 1}}}\left( {{\delta _{{y_i}}} - {\delta _n}} \right)} 
\end{eqnarray}
As the strong duality holds,the primal and dual objectives coincide. Plug Eq. \eqref{eq:opt_beta} into Eq. \eqref{eq:dual}, we have:
\begin{equation*}
\sum\limits_{i,n} {{\eta _{in}}\left[ {1 - {\varepsilon _{n{y_i}}} + {{\hat Y}_{in}}\left( {\beta_n^* } \right) - {{\hat Y}_{i{y_i}}}\left( {\beta_{y_i}^* } \right) - {\xi _i^*}} \right]}=0
\end{equation*}
Expand the equation above, we have:
\begin{eqnarray}\nonumber
\sum\limits_{i,n} {{\eta _{in}}\left[ { {\varepsilon _{n,{y_i}}}-1 + \frac{{\left( {{{\alpha '}_{i{y_i}}} - {{\alpha '}_{in}}} \right)}}{{\psi_{ii}^{ - 1}}} - {\xi _i}} \right]} 
= {\lambda }\sum\limits_r {{{\left\| {\beta _r^*} \right\|}^2}}  \ge 0\nonumber
\end{eqnarray}
Rearranging the above, we obtain:
\begin{eqnarray}\label{eq:link1}
\sum\limits_{i,n} {{\eta _{in}}\left[ { {\varepsilon _{n,{y_i}}} -1+ \frac{{\left( {{{\alpha '}_{i{y_i}}} - {{\alpha '}_{in}}} \right)}}{{\psi_{ii}^{ - 1}}}} \right]}  
 \ge \sum\limits_{i,n} {{\eta _{in}}{\xi _i}}  = \sum\limits_i {{\xi _i}} 
\end{eqnarray}
The left-hand side of Inequation \eqref{eq:link1} can be bounded by:
\begin{eqnarray}
&&\sum\limits_{i,n} {{\eta _{in}}\left[ { {\varepsilon _{n{y_i}}}-1 + \frac{{\left( {{{\alpha '}_{i{y_i}}} - {{\alpha '}_{in}}} \right)}}{{\psi_{ii}^{ - 1}}}} \right]} \nonumber\\ &&\le \sum\limits_i {\left( {\sum\limits_n {{\eta _{in}}\mathop {\max }\limits_r \left\{ { {\varepsilon _{r{y_i}}} -1 + \frac{{\left( {{{\alpha '}_{i{y_i}}} - {{\alpha '}_{ir}}} \right)}}{{\psi_{ii}^{ - 1}}}} \right\}} } \right)}  \nonumber\\
&&= \sum\limits_i {\left( {\sum\limits_n {{\eta _{in}}{{\bar \xi }_i}} } \right)}  = \sum\limits_i {\bar \xi_i }
\end{eqnarray}
\end{proof}

%\section{Results on MNIST and USPS}\label{appd:rs}
%\begin{table*}[h]
\subfloat[10 examples per class]
{\resizebox{0.5\textwidth}{!}{\begin{tabular}{|c|c|c|c|c|c|c|c|}
     \hline 
     & \multicolumn{7}{c|}{Noise Level} \\ \hline
     & 0.0 & 0.2 & 0.3 & 0.5 & 0.8 & 0.9 & 1.0\\ \hline
      SMTLe\_s & \textbf{88.13} & \textbf{86.98} & \textbf{86.62} & \textbf{85.44} & \textbf{84.26} & \textbf{83.61} & \textbf{82.79}\\ 
      SMTLe\_m & 87.29 & 85.20 & 83.81 & 81.47 & 77.74 & 76.45 & 76.26\\ 
      MKT\_{m} & 65.01* & 61.51* & 59.02* & 53.92* & 49.24* & 48.34* & 47.51*\\ 
      MKT\_{s} & 87.82 & 85.88 & 84.00 & 80.85 & 75.87 & 73.81 & 72.30\\ 
      MKT\_{a} & 72.37 & 72.32 & 72.36 & 72.30 & 72.32 & 72.32 & 72.25\\ 
      MKTL & 79.63 & 68.80* & 69.55* & 59.74* & 52.04* & 50.47* & 42.47*\\ 
      Feature+ & 77.60 & 73.33 & 70.73* & 64.90* & 56.86* & 54.40* & 52.3*\\ 
      PMT & 72.86 & 72.87 & 72.87 & 72.87 & 72.86 & 72.86 & 72.87\\ 
      NT & 72.22 & 72.23 & 72.23 & 72.23 & 72.23 & 72.23 & 72.23\\ 
      Batch & 87.46 & 84.78 & 82.96 & 78.04 & 65.96* & 60.97* & 55.65*\\ 
\hline\end{tabular}}}\qquad
\subfloat[15 examples per class]{\resizebox{0.5\textwidth}{!}{\begin{tabular}{|c|c|c|c|c|c|c|c|}
     \hline 
     & \multicolumn{7}{c|}{Noise Level} \\ \hline
     & 0.0 & 0.2 & 0.3 & 0.5 & 0.8 & 0.9 & 1.0\\ \hline
      SMTLe\_s & 88.63 & \textbf{87.52} & \textbf{87.12} & \textbf{86.33} & \textbf{85.57} & \textbf{85.04} & \textbf{84.65}\\ 
      SMTLe\_m & \textbf{88.92} & 86.98 & 85.99 & 84.01 & 80.69 & 80.08 & 79.28\\ 
      MKT\_{m} & 67.03* & 63.34* & 61.31* & 56.26* & 51.76* & 50.72* & 49.69*\\ 
      MKT\_{s} & 88.08 & 86.31 & 85.40 & 83.38 & 79.52 & 77.83 & 77.04\\ 
      MKT\_{a} & 76.19 & 76.16 & 76.19 & 76.12 & 76.14 & 76.16 & 76.11\\ 
      MKTL & 83.75 & 74.27* & 77.46 & 66.39* & 61.57* & 60.0* & 55.28*\\ 
      Feature+ & 81.87 & 78.54 & 76.63 & 72.25* & 64.45* & 61.88* & 59.67*\\ 
      PMT & 76.78 & 76.78 & 76.78 & 76.79 & 76.78 & 76.78 & 76.78\\ 
      NT & 76.09 & 76.10 & 76.10 & 76.10 & 76.10 & 76.10 & 76.10\\ 
      Batch & 87.72 & 85.20 & 83.57 & 79.29 & 68.95* & 64.36* & 59.8*\\ 
\hline\end{tabular}}}\\
\subfloat[20 examples per class]{\resizebox{0.5\textwidth}{!}{\begin{tabular}{|c|c|c|c|c|c|c|c|}
     \hline 
     & \multicolumn{7}{c|}{Noise Level} \\ \hline
     & 0.0 & 0.2 & 0.3 & 0.5 & 0.8 & 0.9 & 1.0\\ \hline
      SMTLe\_s & 88.87 & 87.81 & \textbf{87.28} & \textbf{86.50} & \textbf{85.98} & \textbf{85.64} & \textbf{85.27}\\ 
      SMTLe\_m & \textbf{89.39} & \textbf{87.97} & 87.06 & 85.47 & 82.63 & 81.95 & 81.32\\ 
      MKT\_{m} & 68.03* & 64.37* & 62.7* & 58.98* & 54.41* & 53.52* & 52.93*\\ 
      MKT\_{s} & 88.04 & 86.42 & 85.58 & 83.69 & 81.29 & 80.66 & 79.56\\ 
      MKT\_{a} & 78.68 & 78.64 & 78.65 & 78.62 & 78.63 & 78.63 & 78.60\\ 
      MKTL & 86.75 & 81.14 & 82.46 & 72.57* & 65.4* & 69.53* & 61.56*\\ 
      Feature+ & 83.80 & 80.99 & 79.29 & 75.54* & 68.46* & 66.12* & 63.97*\\ 
      PMT & 78.43* & 78.43* & 78.44* & 78.44* & 78.43* & 78.43* & 78.43*\\ 
      NT & 78.58 & 78.59 & 78.59 & 78.60 & 78.60 & 78.59 & 78.60\\ 
      Batch & 87.80 & 85.41 & 83.89 & 80.10 & 71.22* & 67.35* & 63.28*\\ 
\hline\end{tabular}}}\qquad
\subfloat[25 examples per class]{\resizebox{0.5\textwidth}{!}{\begin{tabular}{|c|c|c|c|c|c|c|c|}
     \hline 
     & \multicolumn{7}{c|}{Noise Level} \\ \hline
     & 0.0 & 0.2 & 0.3 & 0.5 & 0.8 & 0.9 & 1.0\\ \hline
      SMTLe\_s & 89.22 & 88.16 & 87.67 & \textbf{86.94} & \textbf{86.46} & \textbf{86.26} & \textbf{85.95}\\ 
      SMTLe\_m & \textbf{89.69} & \textbf{88.7} & \textbf{87.72} & 86.33 & 83.95 & 83.23 & 82.73\\ 
      MKT\_{m} & 69.15* & 66.04* & 64.23* & 60.71* & 56.0* & 54.9* & 54.53*\\ 
      MKT\_{s} & 88.21 & 86.6 & 85.87 & 84.14 & 82.02 & 81.51 & 81.04\\ 
      MKT\_{a} & 80.35 & 80.33 & 80.36 & 80.33 & 80.33 & 80.33 & 80.31\\ 
      MKTL & 87.50 & 84.39 & 81.88 & 72.97* & 70.57* & 70.17* & 62.88*\\ 
      Feature+ & 85.23 & 82.76 & 81.21 & 77.66* & 71.59* & 69.39* & 67.48*\\ 
      PMT & 79.58* & 79.59* & 79.59* & 79.59* & 79.59* & 79.59* & 79.58*\\ 
      NT & 80.27 & 80.29 & 80.28 & 80.29 & 80.29 & 80.29 & 80.29\\ 
      Batch & 88.02 & 85.80 & 84.33 & 80.92 & 73.36* & 69.99* & 66.27*\\ 
\hline\end{tabular}}}\caption{Results on MNIST with 10/15/20/25 positive examples for each class}\label{tab:mnist}
\end{table*}



\begin{table*}
\subfloat[10 examples per class]{\resizebox{0.5\textwidth}{!}{\begin{tabular}{|c|c|c|c|c|c|c|c|}
     \hline 
     & \multicolumn{7}{c|}{Noise Level} \\ \hline
     & 0.0 & 0.2 & 0.3 & 0.5 & 0.8 & 0.9 & 1.0\\ \hline
      SMTLe\_s & \textbf{91.12} & 89.79 & 89.23 & 88.06 & 86.22 & 85.33 & 84.60\\ 
      SMTLe\_m & 90.78 & \textbf{89.85} & \textbf{89.42} & \textbf{88.55} & \textbf{86.80} & \textbf{85.96} & \textbf{84.91}\\ 
      MKT\_{m} & 86.80 & 84.80 & 83.57 & 81.02 & 75.96 & 74.53* & 72.73*\\ 
      MKT\_{s} & 64.18* & 61.39* & 61.80* & 62.93* & 64.85* & 65.29* & 65.55*\\ 
      MKT\_{a} & 75.76 & 75.76 & 75.75 & 75.79 & 75.75* & 75.78 & 75.84\\ 
      MKTL & 90.24 & 88.13 & 86.20 & 86.07 & 81.82 & 80.18 & 80.14\\ 
      Feature+ & 88.42 & 86.56 & 85.28 & 83.23 & 79.79 & 78.68 & 77.17\\ 
      PMT & 75.89 & 75.89 & 75.90 & 75.88 & 75.88 & 75.88 & 75.87\\ 
      NT & 75.75 & 75.75 & 75.74 & 75.76 & 75.76 & 75.76 & 75.74\\ 
      Batch & 91.65 & 90.38 & 89.58 & 87.36 & 82.55 & 80.52 & 77.84\\ 
\hline\end{tabular}}} \qquad
\subfloat[15 examples per class]{\resizebox{0.5\textwidth}{!}{\begin{tabular}{|c|c|c|c|c|c|c|c|}
     \hline 
     & \multicolumn{7}{c|}{Noise Level} \\ \hline
     & 0.0 & 0.2 & 0.3 & 0.5 & 0.8 & 0.9 & 1.0\\ \hline
      SMTLe\_s & \textbf{91.58} & 90.58 & 89.84 & 88.82 & 86.82 & 86.28 & 85.45\\ 
      SMTLe\_m & 91.46 & \textbf{90.72} & \textbf{90.29} & \textbf{89.61} & \textbf{88.22} & \textbf{87.89} & \textbf{87.19}\\ 
      MKT\_{m} & 89.06 & 87.98 & 87.31 & 85.46 & 81.94 & 80.29 & 78.51*\\ 
      MKT\_{s} & 74.12* & 71.44* & 71.63* & 72.16* & 73.36* & 73.56* & 73.56*\\ 
      MKT\_{a} & 79.57 & 79.57 & 79.55 & 79.58 & 79.57* & 79.59 & 79.58\\ 
      MKTL & 88.74 & 89.45 & 88.86 & 87.63 & 84.53 & 82.30 & 84.41\\ 
      Feature+ & 89.98 & 88.6 & 87.64 & 85.97 & 83.30 & 82.36 & 81.00\\ 
      PMT & 79.56 & 79.54* & 79.54 & 79.55* & 79.56* & 79.55* & 79.55\\ 
      NT & 79.55 & 79.56 & 79.54 & 79.57 & 79.57 & 79.58 & 79.54\\ 
      Batch & 91.75 & 90.61 & 89.88 & 88.04 & 84.25 & 82.69 & 80.80\\ 
\hline\end{tabular}}}\\
\subfloat[20 examples per class]{\resizebox{0.5\textwidth}{!}{\begin{tabular}{|c|c|c|c|c|c|c|c|}
     \hline 
     & \multicolumn{7}{c|}{Noise Level} \\ \hline
     & 0.0 & 0.2 & 0.3 & 0.5 & 0.8 & 0.9 & 1.0\\ \hline
      SMTLe\_s & 91.95 & 91.18 & 90.64 & 89.55 & 87.80 & 87.16 & 86.46\\ 
      SMTLe\_m & \textbf{92.01} & \textbf{91.25} & \textbf{90.79} & \textbf{90.20} & \textbf{89.03} & \textbf{88.70} & \textbf{88.30}\\ 
      MKT\_{m} & 89.90 & 89.05 & 88.64 & 87.12 & 82.78 & 81.21* & 79.44*\\ 
      MKT\_{s} & 79.39* & 76.88* & 76.84* & 77.31* & 78.02* & 78.13* & 78.03*\\ 
      MKT\_{a} & 81.92 & 81.92 & 81.92 & 81.93 & 81.94* & 81.94 & 81.91\\ 
      MKTL & 90.67 & 89.08 & 89.31 & 88.84 & 85.26 & 85.64 & 85.03\\ 
      Feature+ & 90.95 & 89.66 & 88.89 & 87.49 & 85.03 & 84.14 & 83.06\\ 
      PMT & 81.49* & 81.48* & 81.48* & 81.50* & 81.51* & 81.49* & 81.50*\\ 
      NT & 81.89 & 81.92 & 81.87 & 81.91 & 81.94 & 81.93 & 81.89\\ 
      Batch & 91.91 & 90.85 & 90.20 & 88.55 & 85.40 & 84.21 & 82.76\\ 
\hline\end{tabular}}}\qquad
\subfloat[25 examples per class]{\resizebox{0.5\textwidth}{!}{\begin{tabular}{|c|c|c|c|c|c|c|c|}
     \hline 
     & \multicolumn{7}{c|}{Noise Level} \\ \hline
     & 0.0 & 0.2 & 0.3 & 0.5 & 0.8 & 0.9 & 1.0\\ \hline
      SMTLe\_s & 92.18 & 91.43 & 90.93 & 89.97 & 88.41 & 87.83 & 87.22\\ 
      SMTLe\_m & \textbf{92.27} & \textbf{91.68} & \textbf{91.29} & \textbf{90.62} & \textbf{89.58} & \textbf{89.17} & \textbf{88.81}\\ 
      MKT\_{m} & 90.35 & 89.67 & 89.35 & 88.08 & 84.91 & 83.27* & 81.37*\\ 
      MKT\_{s} & 82.40* & 80.24* & 80.12* & 80.46* & 80.93* & 80.98* & 80.83*\\ 
      MKT\_{a} & 83.69 & 83.70 & 83.66 & 83.70 & 83.71 & 83.72 & 83.69\\ 
      MKTL & 91.55 & 90.46 & 90.01 & 88.49 & 87.36 & 86.47 & 87.02\\ 
      Feature+ & 91.38 & 90.26 & 89.56 & 88.38 & 86.42 & 85.63 & 84.67\\ 
      PMT & 82.89* & 82.87* & 82.88* & 82.88* & 82.89* & 82.88* & 82.9*\\ 
      NT & 83.66 & 83.69 & 83.65 & 83.68 & 83.71 & 83.70 & 83.65\\ 
      Batch & 92.11 & 91.08 & 90.50 & 89.06 & 86.34 & 85.30 & 84.27\\ 
\hline\end{tabular}}}\caption{Results on USPS with 10/15/20/25 positive examples for each class} \label{tab:usps}
\end{table*}




\bibliographystyle{splncs03}
\bibliography{research}


%\bibauthoryear

\end{document}
